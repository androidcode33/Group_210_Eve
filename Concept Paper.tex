\documentclass[11pt]{article}
\usepackage{setspace}
\usepackage{graphicx}
%\usepackage{tgbonum}
\usepackage{fullpage}
\usepackage{array}
\onehalfspacing
\begin{document}	
\title{SCHOOL OF COMPUTING AND INFORMATICS\\ TECHNOLOGY}
\author{GROUP 210 EVE}
\date{\today{}}
\begin{figure}
	\begin{center}
	\Huge MAKERERE \includegraphics[width=172pt]{muk.png} \Huge UNIVERSITY
	\end{center}
\end{figure}
\pagenumbering{gobble}
	\maketitle
	
	\begin{center}
	RESEARCH METHODOLOGY \\CONCEPT PAPER
	\end{center}
\begin{center}
	\title{ON}
\end{center}
	\begin{center}
		\title{COMPUTER VISION FOR PHYSICAL SECURITY}\\		 
	\end{center}

   \begin{center}
   	 \title{GROUP MEMBERS}
   	\begin{table}[!th]
   		\begin{tabular}{|l|c|r|}
   			\hline
   			NAME & REGISTRATION NUMBER & STUDENT NUMBER \\
   			MUHWEZI JERALD & 14/U/25199 & 214024819 \\
   			NDAGANO ROBERT & 13/U/22514/EVE & STUDENT NO \\
   			NAMULI GRACE & 14/U/12296/EVE & STUDENT NO \\
   			EKWARO DOMINIC & 15/U/260 & STUDENT NO \\
   			\hline
   		\end{tabular}
   	\end{table}
   \end{center}
     
	\pagenumbering{arabic}
	\newpage
	\section{ \textbf{Introduction} }
	 \paragraph{\textmd{Computer vision is an interdisciplinary field that deals with how computers can be made for gaining high-level understanding from digital images or videos. From the perspective of science and engineering, it pursues to automate tasks that the human visual system can fix.\cite{DUMMY:4}
	And in the Sub-domains of computer vision in relation to security include; scene reconstruction, event detection, video tracking, object recognition, object pose estimation, motion estimation, and image restoration among others.\\
	Physical security is often a second thought when it comes to information security and since physical security has technical and administrative elements, it is oftenly overlooked because most organizations focus on “technology-oriented security countermeasures” \cite{DUMMY:2} to prevent hacking attacks.
	The computer vision system will be designed for the protection of personnel, hardware, software, networks and data from physical actions and events that could cause serious loss or damage to an enterprise or company.}}
	 
	 \subsection{\textbf{Background to the problem}}
	  \paragraph{\textmd{Computer vision as a discipline that has made
	 a significant impact on a number of diverse application domains eg \cite{DUMMY:1} and it has been around since the 1960s.\\ Beginning from the seventies through the nineties, computer vision started proving its practical value in a wide range of diverse application domains including medical diagnostics,manufacturing, environmental monitoring, space exploration, and military systems such as automatic target recognition, precision weapons, reconnaissance\cite{DUMMY:3} and currently its at an extraordinary point in its development.\\Physical security over  the past decades has become increasingly more difficult for organizations. Technology and computer environments now allow more compromises to occur
	 due to increased vulnerabilities. USB hard drives, laptops, tablets and smartphones allow for information to be lost or stolen because of portability and mobile access. In the early days of
	 computers, they were large mainframe computers only used by a few people and were secured in locked rooms \cite{DUMMY:2}. Today, desks are filled with desktop computers and mobile laptops that have access to company data from across the enterprise. Protecting data, networks and systems has become difficult to implement with mobile users being able to take their computers out of the facilities. Fraud, vandalism, sabotage, accidents, and theft are increasing costs for organizations since the environments are becoming more “complex and dynamic” \cite{DUMMY:2}.}}
	  
	 \subsection{\textbf{Problem Statement}}
	 
	 \paragraph{\textmd{limited memory-cannot remember a quickly flashed image. 
	 limited to visible spectrum (The visible spectrum is the portion of the electromagnetic spectrum that is visible to the human eye. Electromagnetic radiation in this range of wavelengths is called visible light or simply light.)	
	 illusion (thing that is or is likely to be wrongly perceived or interpreted by the senses.)}}
	 
	 \subsection{\textbf{purpose}}
	 
	   \paragraph{\textmd{To develop method that enable a machine to understand, analyze, process and acquire digital images, videos and extraction of high-dimensional data from the real world in order to produce numerical or symbolic information that is used for the protection of personnel, hardware, software, networks and data from physical actions and events that could cause serious loss or damage to an enterprise, agency or institution.}}
	   
	   \subsubsection{\textbf{ Objectives}}
	   
	   \begin{enumerate}
	   
	   \item To collect data on the current existing systems
	   \item To analyze the data collected and generate requirements
	   \item To design and implement the proposed system 
	   \item To test and validate the system.
	           
	   \end{enumerate} 	  

	   
	   
	   
	   \section{\textbf{Methodology}}
	   
	   		\begin{itemize}
	   	\item	To achieve the first objective. We shall carry-out literature review related systems where computer vision is being applied. we shall also carry-out interviews, to gather vital information to aid designing of the proposed system
	   	\item	Secondly, we intend to use excel software to analyze the collected data. Models such as Data Flow Diagrams (DFD) will be used to document and visual the true and real requirements for the system being developed.
	   	\item	Thirdly, we shall use ERD and UML diagrams to design the system. We intend to use machine learning and python for implementing the system.
	   	\item	Lastly, we shall test the system in order to correct errors or remove defects that will have arose through compiling and running on the development platforms.
	   		\end{itemize}
	   		
    	\bibliography{refelences0007}
    	\bibliographystyle{apalike} 
       
\end{document}
