\documentclass[11pt]{article}
\usepackage{setspace}
\usepackage{graphicx}
%\usepackage{tgbonum}
\usepackage{fullpage}
\usepackage{array}
\onehalfspacing
\begin{document}

		
\title{SCHOOL OF COMPUTING AND INFORMATICS\\ TECHNOLOGY}
\author{GROUP 210 EVE}
\date{\today{}}
\begin{figure}
	\begin{center}
	\Huge MAKERERE \includegraphics[width=172pt]{muk.png} \Huge UNIVERSITY
	\end{center}
\end{figure}
\pagenumbering{gobble}
	\maketitle
	
	\begin{center}
	RESEARCH METHODOLOGY \\CONCEPT PAPER
	\end{center}
\begin{center}
	\title{ON}
\end{center}
	\begin{center}
		\title{COMPUTER VISION FOR PHYSICAL SECURITY}
	\end{center}
    
	\newpage
	\tableofcontents
	
	\pagenumbering{arabic}
	\newpage
	\section{ \textbf{Introduction} }
	 \paragraph{\textmd{An introduction should announce your topic, provide context and a rationale for your work, before stating your research questions and hypothesis. Well-written introductions set the tone for the paper, catch the reader's interest, and communicate the hypothesis or thesis statement. \emph{computer vision for physical security}.}}
	 
	 \subsection{\textbf{Background to the problem}}
	  \begin{itemize}
	 	
	 \item Describe the question, problem or need that needs to be addressed. Briefly provide supporting documentation for the importance of addressing this question, problem or need. 
	 \item You may include supporting statistical data in this section, but it should be brief. Numbers are always convincing. \cite{DUMMY:2}
	 \item Indicate why anyone should care. It is easy to forget that everyone does not understand the situation as well as you, especially when you are so close to a particular topic.
	 \item Make sure that you cite or refer to what others have accomplished relative to your project or research so that you convince the reader that you are an expert on this particular issue and more needs to be done. 
	\item Beware of stating that you are the only person who has ever proposed such a project/idea. Even the most innovative concepts are based on the work of others from related fields.
	 	
	 \end{itemize} 
	 
	 \subsection{\textbf{Problem Statement}}
	 
	 \paragraph{\textmd{provides the purpose for the research.  This section of the Concept Paper introduces the problem under investigation, addresses why the researcher wants to investigate this problem, and how the research findings may help.  Supporting documentation, including statistical data if available, should be used to emphasize the need for this research.  This section is one of the most important sections of the Concept Paper; its serves to gain the reader’s attention and support.}}
	 
	 \subsection{\textbf{Objectives}}
	 
	   \paragraph{\textmd{To develop methods that enable a machine to understand, analyze, process and acquire digital images, videos and extraction of high-dimensional data from the real world in order to produce numerical or symbolic information that is used for the protection of personnel, hardware, software, networks and data from physical actions and events that could cause serious loss or damage to an enterprise, agency or institution.}}
	   
	   \subsubsection{\textbf{Specific Objectives}}
	   
	   \begin{enumerate}
	   
	   \item Specific objective one
	   \item Specific objective two
	   \item Specific objective three
	   \item Specific objective four
	           
	   \end{enumerate} 
	   \subsection{\textbf{Research Scope}}

	   \subsection{\textbf{Research Significance}}
	   \subsection{\textbf{Literature Review}}
	   
	   \section{\textbf{Methodology}}
	   \paragraph{\textmd{provides the student’s best idea on how to conduct the research and analyze the data. The goals and objects identified in previous sections of the Concept Paper should relate to the research methods described in this section.  For the Concept Paper, the methodology is simplified or summarized as discussed by eg \cite{DUMMY:1}}}
	   
	
    	\bibliography{refelences0007}
    	\bibliographystyle{apalike}
    	
    	\newpage
      \LARGE {Tips for writing a successful concept paper:}
       \begin{itemize}
       	\item Try to be brief, concise, and clear. Concept papers should not be longer than 3 pages. 
       	\item Don’t overwhelm the reader with detail, but avoid sounding vague or unsure about what you want to accomplish. 
       	\item Be positive and definite. Instead of saying an objective “may be accomplished,”indicate that the objective “will be accomplished” by a certain time. 
       	\item Consider your audience. If your concept paper is going to be reviewed by those in your field, scientific terms and technical jargon may be acceptable. However, if your proposal is being reviewed by generalists or those outside your field, this type of language will not communicate your ideas effectively.
       \end{itemize}
       	
       
       
\end{document}